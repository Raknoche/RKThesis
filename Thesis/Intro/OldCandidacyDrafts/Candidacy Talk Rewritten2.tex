\documentclass[a4paper,12pt]{article}


\usepackage{amsmath}
\usepackage{graphicx}
\usepackage[capposition=bottom]{floatrow}
\usepackage{amssymb}

\usepackage{caption}

\begin{document}
\title{Triated Methane as an Internal Calibration Source in Xenon Detectors}
\author{Richard Knoche}
\maketitle

\begin{abstract}
This paper provides an introduction to the cosmological evidence for dark matter, and briefly discusses the particle physics theories which seek to explain this evidence.  It then discusses terrestrial WIMP detection experiments with a focus on the Large Underground Xenon detector (LUX).  Details of the LUX experiment's effort to develop an internal calibration source for use in xenon detectors are included, and the results of R\&D projects to meet this goal are presented.
\end{abstract}

\tableofcontents



\section{Introduction to Dark Matter}

In recent years it has been discovered that the luminous matter which scientists have studied for centuries is only a small fraction of the total composition of the universe.  There is strong evidence of dark baryonic matter (a few percent), nonbaryonic hot dark matter ($\sim$0.1\%-1.5\%), non baryonic cold dark matter ($\sim$25\%), and dark energy ($\sim$70\%) components to the universe, where the percentages are as a total fraction of their contribution to the universe's total composition.  While we know a relatively large amount about the luminous matter which contributes $ \lesssim $ 1\% of the universe's total composition, we know very little about these larger components.  In particular, while we understand certain characteristics of the cold dark matter component, there is no consensus on what it consists of.  Before examining the experiments which seek to answer this question we will first discuss what is currently known about the nonbaryonic dark matter component.

\subsection{Evidence of Dark Matter}

In the early 1930's Fritz Zwicky measured the rotational velocity of eight galaxies in the Coma Cluster.  Newtonian physics predicts that these rotational velocities will be given by $v(r)=\sqrt{GM(r)/r}$, where $M(r)$ is the mass enclosed by the orbit and G is Newton's gravitational constant.  In his examination, Zwicky found that the velocities at large radii were too high to be consistent with the Newtonian prediction arising from the visible matter alone. [1] This discrepancy was reinforced in the 1970's, when further data on the rotational velocity of spiral galaxies began to be taken.  Instead of the rotational velocity falling off as $\propto 1/\sqrt{r}$ beyond the radius of visible matter as one would expect, the rotational velocity rose for small radii, then asymptoted to a constant $ v \simeq 100-300 km/s $ [2,3,4] for large radii in most galaxies.  The most widely accepted explanation of this phenomenon is that the disk galaxies are immersed in a dark matter (DM) halo such that $ M(r)/r $ remains constant at large radii.  Such a halo could form from a self gravitating ball of ideal gas at a uniform temperature.

\begin{center}
\includegraphics[scale=1]{RotationCurve.png}
\captionof{figure}{Rotation curve of a spiral galaxy.  The blue line (A) indicates the rotational velocity predicted by Newtonian physics when luminous matter alone is taken into acount.  The red line (B) indicates true rotational velocity seen from observations.}
\end{center}

To further examine the properties of these dark matter halos it is useful to introduce a quantitative measure for the composition of the universe.  Beginning with the Friedmann equations, assuming the cosmological constant $\Lambda = 0$ and setting the normalized spatial curvature $k = 0$ one can find the critical density for which the spatial geometry of the universe is Euclidean to be
\[\rho_c =\frac{3H_0^2}{8 \pi G} \]
where $H_0$ is the present value of the Hubble constant and G is Newton's gravitational constant. [5] The current experimental value for $H_0$ in the dimensionless units 100 km/s/Mpc is $ h \sim 0.7 $ with an uncertainty of $\sim5\%$.  We can then define the density parameter as 
\[\Omega=\frac{\rho}{\rho_c}=\frac{8 \pi G \rho}{3 H_0^2}.\]
If $\Omega$ is larger than unity the universe is spatially closed, and if $\Omega$ is less than unity the universe is spatially open. This density parameter can be split into components, such that for a particular component $x$
\[\Omega_x = \frac{\rho_x}{\rho_c}.\]

Detailed cosmological studies have concluded that all the luminous matter in the universe has a density parameter of $\Omega_{lum} \lesssim 0.01$.  This information, combined  with the fact that analysis of galactic rotational velocities implies $> $90\% of the mass in galaxies is dark leads to the conclusion that $\Omega_{DM} \geq 0.1$.  This is only a lower limit on the dark matter density parameter, since most rotation curves remain flat out to the largest radii at which they can be measured and it can be assumed that the DM halos extend even further out.


With $\Omega_{DM} \geq 0.1$ it is possible that baryonic DM alone could be responsible for the dark halos.  However, other analyses eliminate this possibility. Direct searches for massive compact halo objects (MACHOs) utilizing microlensing have determined that $<$25\% of the dark halos could be due to baryonic dark matter within the mass range of $ 2 \times 10^{-7} M_{sun} < M < 1 M_{sun} $ at a 95\% confidence limit. [6,7] Furthermore, data from the Hubble Deep Field Space Telescope suggests dark matter halos consist of $\leq$5\% white dwarfs.

With baryonic dark matter being ruled out as the sole component of dark matter halos we now investigate the other density parameter components. Big Bang nucleosynthesis models constrain the amount of baryonic matter in the universe to $\Omega_b \approx 0.045$ (where b stands for baryons). [8]  Additionally, analysis of velocity flows, x-ray emissions temperatures, and gravitational lensing in large clusters and superclusters of galaxies suggests that the total matter component of the universe has density parameter $\Omega_m \approx 0.2-0.3$.  One can combine this information, assuming $h=0.7$ to find (with 1 $\sigma$ errors) that
\[\Omega_b = 4.6 \pm 0.1\% \]
\[\Omega_{nbm} = 22 \pm 2\% \]
\[\Omega_\Lambda =73 \pm 4\% \]
where $\Omega_b$ is the baryonic density of the universe, $\Omega_{nbm}$ is the nonbaryonic density parameter of the universe, and $\Omega_\Lambda$ is the dark energy density parameter of the universe. [9] This is known as the $\Lambda$-CDM model.  It should be noted that a greater range of allowed values arise when the model assumptions are varied. 

\subsection{Nonbaryonic Dark Matter}

With $\Omega_{nbm} = 22 \pm 2 \% $ it is intriguing to look at the particles which have been proposed to explain this contribution to the total density parameter. One such particle is the standard-model neutrino.  The neutrino is an electrically neutral, weakly interacting particle with a nearly zero mass.  Neutrinos exist in three distinct flavors -- the electron neutrino ($\nu_e$), the muon neutrino ($\nu_\mu$), and the tau neutrino ($\nu_\tau$).  It is known that neutrinos oscillate between these three flavors, with each flavor state being a super position of three neutrino states of definite mass ($\nu_1$,$\nu_2$, and $\nu_3$).  Experiments studying solar neutrino oscillations have determined the squared mass difference between what is known as the solar neutrino doublet ($\nu_1$ and $\nu_2$) to be $\delta m^2 = (7.66 \pm 0.35) \times 10^{-5} eV^2$, while experiments studying atmospheric neutrino oscillations have determined the remaining squared mass difference between the solar neutrino doublet and $\nu_3$ to be $\pm (2.38 \pm 0.27) \times 10^{-3} eV^2$ up to an unknown sign. [10]  This sign ambiguity leads to two possibly hierarchies for the neutrino mass states. (Figure 2)  If we assume the normal hierarchy to be true, we can set a lower limit on the most massive neutrino state to be $ m_{\nu_3} \gtrsim 0.05$ eV.

\begin{center}
\includegraphics[scale=0.4]{neutrinos.png}
\captionof{figure}{The two hierarchies of neutrino mass states.  Black, teal, and red indicated the three flavors of neutrinos, while one, two, and three indicated the three mass states. Reproduced from Reference (11).}
\end{center}


The density parameter of neutrinos is given by
\[\Omega_\nu=\frac{\rho_\nu}{\rho_c}=\frac{1}{h^2}\sum_{i=1}^3 \frac{g_i m_i}{90eV},\]
where $g_i = 1$ for Majorana neutrinos (own antiparticle) and $g_i = 2$ for Dirac neutrinos (distinct antiparticles). [12] Using the lower mass limit of the neutrino and assuming Majorana neutrinos, this suggests a lower limit on the neutrino density parameter of $\Omega_\nu \gtrsim 0.00122$.  Thus, neutrinos do provide some contribution to the nonbaryonic dark matter density parameter.  

To find an upper limit on the neutrino contribution to the nonbaryonic dark matter density parameter is is necessary to distinguish hot dark matter from cold dark matter.  Hot dark matter is composed of particles that have zero or nearly-zero mass.  Special relativity requires that the massless particles move at the speed of light, and that the nearly-massless particles move close to the speed of light.  As a result hot dark matter forms very hot gases.  Cold dark matter is composed of particles that have sufficient mass to travel at sub-relativistic velocities, thus forming colder gases.  With their low masses neutrinos fall under the hot dark matter category.  A combination of galaxy clustering measurements, CMB observations, and Lyman-$\alpha$ observations give an upper limit on the hot dark matter contribution of $\Omega_\nu \lesssim 0.0155$, thus neutrinos and other hot dark matter particles can not be the primary contribution to the nonbaryonic dark matter density parameter. [9]

If we assume cold dark matter (CDM) particles were in thermal equilibrium with the other standard-model particles during the early stages ($<$1 ns) of the universe it is possible to calculate the CDM density parameter.  According to Maxwell-Boltzmann statistics, as the temperature, $T$, of the universe cools, the particles with masses $m > T$ will diminish exponentially.  Once the temperature of the universe cooled below the CDM mass scale the creation of these particles would have ceased.  At this time the CDM particles which still existed would have continued annihilating with one another.  As time went on, CDM annihilation became less and less likely due to their dwindling abundance.  Once the expansion rate of the universe, given by Hubble's constant, exceeded the CDM annihilation/creation rate, the CDM particles dropped out of thermal equilibrium and the CDM density became fixed.  

The density parameter for CDM is approximately given by
\[\Omega_\chi h^2 \simeq \frac{T_0^3}{M_{Pl} \langle \sigma_A \nu \rangle} \simeq \frac{0.1 \; pb \; c}{\langle \sigma_A \nu \rangle}, \]
where $\sigma_A$ is the total annihilation cross section of CDM particles, $\nu$ is the relative velocity of CDM particles, $T_0$ is the equilibrium temperature, $M_{Pl}$ is the Planck mass, $c$ is the speed of light, and $\langle ... \rangle$ represents an average over the thermal distribution of CDM particle velocities. [13,14] Remarkably, for the total density parameter of the universe to equal unity, as required by cosmological models, a cross section on the order of particles interacting on the electroweak scale is required for CDM particles. This result is the main motivation behind suspecting weakly interacting massive particles (WIMPs) as the dominant contribution to the nonbaryonic dark matter density parameter.  These WIMPs are theorized to have a mass ranging from $\sim$ 10 GeV to a few TeV with an interaction cross sections on the order of the electroweak scale.  

\subsection{WIMPs and SUSY}

Supersymmetry (SUSY) is a symmetry of space-time which has been proposed in an effort to unify the electroweak, strong, and gravitional forces.  This theory offers some insight into the nature of WIMPs.  SUSY requires that a supersymmetric partner particle exist for each particle in the standard model.  These partners go by the names of sleptons (partners of leptons), squarks (partners of quarks), gauginos (partners of gauge bosons), and higgsinos (partners of Higgs bosons).  Sleptons and squarks have spin zero, while gauginos and higgsinos have spin one-half.  Since none of these supersymmetric particles have been discovered it is thought they are far more massive than their standard model counterparts, and thus that supersymmetry is not an explicit symmetry of nature.
  
Goldberg [15] and Ellis [16] have suggested that neutral gauginos and neutral higgsinos can mix together in a superposition known as the neutralino, $\chi$.  In most SUSY models, the neutralino is the lightest supersymmetric particle (LSP).  In models which conserve R-parity (a new quantum number distinguishing SUSY particles from standard model particles) the LSP is stable, making it a prime candidate particle for dark matter.  The expect cross section of neutralinos interacting via inelastic collisions with nucleons is dependent on the allowed regions of parameter space in the SUSY model being used.  Results from the WMAP satellite [9,17] refined this parameter space, concluding that the density parameter of neutralinos would be  $0.192 < \Omega_\chi < 0.263$.

\subsection{Other Dark Matter Candidates}

The neutralino is one of many candidate particles suggested for WIMPs, and as previously mentioned, WIMPs are not the only candidate for dark matter.  In this section we will briefly discuss some of the other candidates.  Figure three shows some of these.  

\begin{center}
\includegraphics[scale=0.7]{DMCandidates.png}
\captionof{figure}{Estimated interaction strengths ($\sigma_{int}$) of some of the dark matter candidates with masses $m_\chi$. Reproduced from Reference (18).}
\end{center}

While we have already discussed the neutrino and WIMPs as dark matter candidates, we have not introduced  axions, axinos, gravitinos, or WIMPzillas.  To do this, we first need a brief introduction to quantum chromodynamics.  Quantumn chromodynamics (QCD) is a theory describing the strong interaction between quarks and gluons, which make up hadrons.  In particle physics there exists a proposed symmetry of nature referred to as charge conjugation parity symmetry (CP-Symmetry).  CP-Symmetry postulates that particles should behave the same if they are replace by their own antiparticle (C symmetry), and then have their parity reversed (P symmetry). Within QCD there is no theoretical reason to assume CP-symmetry exists.  However, when a CP-violation term is included in the QCD lagrangian its coefficient has been experimentally determined to be less than $10^{-10}$. [19] This unexpected result is known as the strong CP problem in quantum chromodynamics.    To reconcile this, a new symmetry known as the Peccei-Quinn theory has been proposed.  This theory postulates the existence of a new pseudoscalar particle called the axion.  According to the Peccei-Quinn theory, axions would be electrically neutral, low mass ( $1\mu eV - 1 eV$) particles which have very low interaction cross-sections for the strong and weak forces.  The axino arises when SUSY is introduced to the Peccei-Quinn theory, resulting in a supersymmetric partner to the axion known as the axino.

In quantum field theory, the graviton is a hypothetical  elementary particle which mediates the gravitational force.  As with axions and axinos, when SUSY is introduced to quantum field theory a supersymmetric partner to the graviton is predicted to exist known as the gravitino.  In some models, gravitinos are the LSP in SUSY and are thus a candidate particle for dark matter.

The final dark matter candidate shown in figure three is the WIMPzilla.  WIMPzillas are supermassive dark matter particles which arise when one considers the possibility that dark matter might be composed of nonthermal supermassive states. These particles would have a mass many order of magnitude higher than the weak scale. [20] Studies have shown that for stable particles with masses close to $10^{13} \; GeV$  WIMPzillas would be produced in sufficient abundance to give $\Omega \approx 1$ for the total density parameter of the universe.  It should be noted that figure three and the discussion of this section does not encompass all of the alternatives to WIMPs.   Although these other dark matter candidates offer intriguing explanations to the dark matter problem, we will now return to our discussion of WIMPs and direct WIMP detection experiments.


\section{Direct Detection of WIMPs}

\subsection{WIMP Recoil Spectrum}
The energy recoil spectrum of WIMPs in terrestrial detectors is given by
\[\frac{dN}{dE_r}=\frac{\sigma_0 \rho_\chi}{2 \mu^2 m_\chi} F^2(q) \int_{v_{min}}^{v_{esc}} \frac{f(v)}{v} dv, \]
where $\rho_\chi$ is the local WIMP density, $f(v)$ is the velocity distribution of WIMPs in the halo, $v_{min}$ is the minimum WIMP velocity able to generate a recoil of energy $E_r$, $v_{esc}$ is the escape velocity for WIMPs in the halo, $F(q)^2$ is the nuclear form factor, $\sigma_0$ is the WIMP-nucleus interaction cross sections, and $\mu$ is the WIMP-nucleus reduced mass given by 
\[\mu=\frac{m_\chi m_N}{m_\chi + m_N}\]
where $m_N$ is the target nucleus mass. [14,21,22] The WIMP-nucleus cross section has a dominant spin-independent term which increases with the atomic number of the target nucleus as $A^2$.  A calculation of both the differential and integrated WIMP event rates in single isotope targets of  $^{131}$Xe, $^{73}$Ge, and $^{40}$Ar using a WIMP mass of 100 GeV from is included in figure four.

\begin{center}
\includegraphics[scale=0.5]{Recoil-spectrum.png}
\captionof{figure}{Calculated differential spectrum in evts/keV/kg/d (solid lines) and the integrated event rate in evts/kg/d (dashed lines)  for $^{131}$Xe, $^{73}$Ge, and $^{40}$Ar assuming a 100 GeV WIMP with spin-indepdent cross section for a WIMP-nucleon of $\sigma=5 \times 10^{-43} cm^2$.}  
\end{center}

Lighter target nuclei will produce lower event rates in a WIMP detector due to their lower cross sections and less effective transfer of energy during nuclear recoil events. While heavier target nuclei produce stronger interaction cross sections, they also result in reduced event rates at high energies due to a loss of coherence from form factor suppression.  To maximize efficiency a xenon detector with a low analysis threshold is ideal.

\subsection{Quenching Factors and Event Discrimination}
In WIMP detectors there will be two types of events.  WIMPs and neutrons will deposit energy via nuclear recoils, while gamma rays and x-rays will deposit energy via electron recoils.  The key to identifying WIMP events within a detector lies in understanding how it will respond to each of these events.  To achieve this goal, a quantity known as the quenching factor of a detector must be determined.  The quenching factor of a detector describes the difference in the amount of energy measured by the detector between the two types of events.  Electron recoil events are typically measured in units of $keV_{ee}$, which is the amount of energy an electron recoil would require to generate the event.  Similarly, nuclear recoil events are typically measured in units of $keV_r$, which is the amount of energy a nuclear recoil would require to generate the event.  The energy scale for electron recoils can be set using a beta emitting calibration source, while the energy scale for nuclear recoils can be set using a neutron calibration source.  A detector's quenching factor is then given by
\[QF=\frac{E_ee(keV_{ee})}{E_r(keV_{r})}.\]

In two phase (liquid and gas) xenon detectors the quenching factor differs between scintillation and ionization signals produced by interactions within the liquid.  The initial recoil event will produce a scintillation signal (referred to as S1) and charged particles.  Photo-multiplier tubes can be used to measure the scintillation light, while the charged particles can be drifted to an anode located in the gas phase of the detector.  Once the charged particles are accelerating toward the anode in the xenon gas they will create an electron cascade, producing an ionization signal (referred to as S2).  Nuclear recoil events have higher ionization density, leading to a higher recombination probability, resulting in a higher S1 yield than electron recoil events. Thus, the different quenching factors can be used to distinguish the two types of events.


\section{The Large Underground Xenon Detector}

\subsection{Introduction to LUX}
The Large Underground Xenon detector (LUX) is one experiment utilizing a xenon WIMP detector. [24,25] It is a dual phase time projection chamber containing 350 kg of xenon used as the target mass.  The xenon is cooled by a thermosyphon system until it condenses in the detector.  Inside of the xenon space there is one photo-multiplier tube array submerged in the liquid xenon, another photo-multiplier tube array suspended in the xenon gas, and five wire grids producing an electric field to drift charged particles through the detector.  The location of the S2 signal provides the x and y coordinates of the recoil event, while the time between the S1 and S2 signal provides the z coordinate of the recoil event.  This allows the primary event to be localized within one centimeter in all three spacial dimensions.  The xenon space is shielded by a 300 ton water tank which houses additional photo-multiplier tubes for cosmic ray vetoing. 

\begin{center}
\includegraphics[scale=0.5]{lux.jpg}
\captionof{figure}{Image depicting the internals of the LUX detector.}
\end{center}

\subsection{Calibration of the Detector}

In order to distinguish dark matter signals in the detector from background signals the detector's response to nuclear recoil events and electron recoil events must be well understood. (Figure 6) It is common to use an external beta emitter such as cesium-137 to achieve this goal.  However, the xenon in LUX has a strong self-shield characteristic (radiation length $ \approx 2.77 $ cm).  While this is convenient for eliminating background radiation, it makes calibrating the inner most regions of the detector a challenge.

\begin{center}
\includegraphics[scale=0.75]{Recoils.jpg}
\captionof{figure}{A plot of electron recoil events generated by an external calibration source in LUX's predecessor LUX-0.1.  This data is used to determine the electron recoil band shown in red, while data from a neutron source (not shown) is used to determine the nuclear recoil band shown in green }
\end{center}

To overcome this problem LUX is making use of internal calibration sources.  An ideal internal calibration source would need to be a single beta emitter in the energy range of interest ($ <15 $ keV) which can be dissolved into the liquid xenon in the detector.  Furthermore, the source must be made of a material with low electronegativity so that it will not poison the detector's charge drift length.  Similarly the source can not attenuate the UV scintillation light produced by recoil events in the detector.  To achieve a reliable calibration in all regions of the detector the source would need to have a long enough life time to diffuse throughout the entire detector (a few hours).  Finally, there must be a method for removing the source once the calibration has finished.  This could simply mean waiting for the source to decay if its half life is short, or actively purifying the source out of the detector if its half life is long. [26]

\section{Tritium as an Internal Calibration Source}

Tritium meets several of these requirements.  It is a beta emitter with a Q-value of 18 keV that produces a broad spectrum over the entire energy range of interest.  Its 12.3 year half life means that the source will have plenty of time to dissolve uniformly throughout the detector.  However, this long half life serves as a double edged sword.  Since its half life is on the order of a decade one could not simply wait for it to decay away -- it must be actively removed from the detector when the calibration is completed.  To complicate this matter bare tritium sticks to most surfaces, including materials like teflon, polyethylene, and steel which make up the majority of most xenon detectors.

To make removing the tritium from the detector more feasible we have made use of tritiated methane ($ CH_3T $).  Methane is highly inert due to its fully saturated carbon-hydrogen bonds.  It has a diffusion constant in polyethylene that is 10 times smaller than hydrogen, and it does not capture electrons that will be drifting through the detector.  By replacing one of the hydrogen atoms in a methane molecule with tritium we combine the strength of both of these materials, resulting in the ideal internal calibration source.


\subsection{Gas Experiments}

To determine the viability of using tritiated methane as an internal calibration source we built a system to inject tritiated methane into a gaseous xenon environment.  This system consists of three sections.  The first section, the xenon space, contains a xenon purifier which uses hot zirconium to remove the tritiated methane, two xenon storage bottles used to move xenon through the system via cryopumping, and a proportional tube used to detect activity within the xenon space.  The second section is a small transfer system which is used to inject consistent amounts of methane into the xenon space with each injection.  The final section consists of a tritiated methane storage bottle used as the source of injections and a SAES MC1-905F methane purifier to remove unwanted contaminates prior to entering the xenon space.

\begin{center}
\includegraphics[scale=0.5]{gassys.jpg}
\captionof{figure}{Front and back of the tritiated methane injection system for the gas phase experiments at UMD.}
\end{center}

The primary goal of this experiment was to determine how to maximize our purification efficiency.  There are two factors which dominate this process -- flow rate through the purifier and rest time between subsequent purifications.  High flow rates through the purifier can cool the zirconium inside, while inadequate rest time between purifications can lead to build up of methane on the surface of the zirconium beads.  Both of these situations lead to a decrease in purifier efficiency. 

\subsection{Gas Experiments Results}

The first black data point in figure eight is our worst purification efficiency, (96\% +/- 1\%) corresponding to our highest flow rate. (8 SLPM compared to the typical ~0.3 SLPM)  While we were unable to control the flow through our experiment as much as we desired, we are at least able to conclude that exceeding the maximum flow rate suggested for the purifier does significantly decrease purification efficiency.

\begin{center}
\includegraphics[scale=0.5]{Figone.jpg}
\captionof{figure}{Single pass efficiency of the purifier when removing tritiated methane.  The red and blue points indicate data taken by different students, while the black points indicate data for which the procedures were intentionally altered. }
\end{center}

We found that allowing for ample rest time between purifications does significantly increase purification efficiency. Our best purifications were the first data points in each cluster in figure nine.  We were able to obtain efficiencies of 99.99\% when the purifier was resting for three weeks or longer, and obtained efficiencies of 99\% to 99.9\% when the purifier was used on a daily basis.

\begin{center}
\includegraphics[scale=0.5]{Figfour.jpg}
\captionof{figure}{Time-separated clusters of purifications have an upward trend in purification inefficiency.  Each cluster shown is separated in time by at least three weeks.}
\end{center}

\subsection{Liquid Experiments}

With the compeletion of the gaseous xenon experiments, we moved on to test removing triated methane from a liquid xenon environment.  To do this we have made use of our liquid xenon system at the University of Maryland.  The system consists of two main sections, the tritiated methane injection system and the liquid xenon system.  We will first discuss the set up of the tritium injection system, pictured below.

\begin{center}
\includegraphics{UMDIS.png}
\captionof{figure}{The tritium injection system for the liquid phase experiments at UMD.}
\end{center}

The injection system begins at the tritiated methane storage bottle.  This bottle is double valved for safety reasons.  As with the gaseous experiments, we have a SAES MC1-905F methane purifier in series with the storage bottle.  Following the methane purifier there is a series of injection volumes branching off to the left.  These injection volumes are designed to inject the desired amount of tritiated methane into the xenon system.  The last component of the injection system is located above the injection volumes.  This plumbing is used to collect all of the tritiated methane from the injection volume via cyropumping.  After the plumbing has warmed, the xenon circulating outside of the injection system is rerouted through the cryopump plumbing to sweep all of the tritiated methane into the xenon system.

The second section of our system, the liquid xenon system, is pictured below.

\begin{center}
\includegraphics[scale=0.5]{cryo.png}
\captionof{figure}{The liquid xenon system at UMD.}
\end{center}

In the liquid xenon system, a refrigerator cold head cools the plumbing in which the xenon circulates.  The cooled xenon then condenses and drips into a liquid xenon storage vessel.  Inside of the liquid xenon storage vessel are two PMTs that are looking at each other.  Once the vessel is filled both of these PMTs are submerged in the liquid xenon.  Note that this means the system at UMD is a single phase detector, rather than a dual phase detector like LUX.  It should be noted that in the plumbing leading to the liquid xenon system there is a SAES Zirconium getter (pcf4c3r1) used to purify the tritiated methane out of the system when desired.

\subsection{Liquid Experiments Results}

Using the lessons learned from the gaseous xenon experiments we were able to remove over 99.9\% of any tritiated methane that was injected into our liquid xenon system.  To illustrate this point, the spectra seen by our PMTs before an injection, after an injection, and after purification are shown in figure twelve.

\begin{center}
\includegraphics[scale=0.5]{spectra.png}
\captionof{figure}{Left: Overlay of spectra seen by PMTs in the liquid xenon detector.  The blue spectrum is what is seen by the PMTs prior to injecting tritium, the red spectrum is what is seen by the PMTs after injecting tritium, and the green spectrum is what is seen by the PMTs after purifying the xenon to remove any injected tritiated methane.  Right: The difference between the before injection and after purification spectra. }
\end{center}


We were able to repeat the injection and purification procedures many times without seeing any rise in the residual background of our system.  Figure thirteen illustrates this point.  So far we have injected over ten thousand becquerel of activity without seeing any rise in residual background rates.

\begin{center}
\includegraphics[scale=0.5]{backgrounds.png}
\captionof{figure}{Left: Time histogram of three of our tritiated methane injections. The y-axis of this figure is count rate, while the x-axis is time.  Right:  Residual background rates over time in our detector after purifying the tritiated methane out of the xenon.  }
\end{center}

\section{Diffusion Experiments}

Will write up at end of current experiments

\section{Tritium Injection System for LUX}

With the knowledge we obtained from our experiments at UMD we have designed and built a tritium injection system for the purpose of calibrating the LUX detector.  This tritium injection system consists of a tritiated methane bottle, a methane purifier, and subsequent expansion volumes to fine tune the amount of activity being injected.  The tritiated methane bottle contains a mixture of tritiated methane and dekryptonated xenon gas.  The dekyrpotonated gas is included in the bottle to raise its pressure and ensure no backward flow from the LUX system into the bottle.

PICTURE OF INJECTION SYSTEM, AND MAYBE PLUMBING DIAGRAM (Dont have this picture yet)

To use the system an operator must first determine which expansion volumes are required to obtain the desired injection activity.  Once these injection volumes have been filled with tritiated methane, the tritiated methane bottle is isolated and the LUX xenon gas flow is diverted through the expansion volumes.  This sweeps the tritiated methane out of the injection system and into the LUX detector.  Since LUX's standard operating mode has a xenon purifier included in its flow path this tritiated methane will be removed from the system as it circulates. 

\begin{thebibliography}{1}

\bibitem{Zwicky} Zwicky.  \emph{Helv. Phys. Acta} 6:110 (1933)
\bibitem{Persic} Persic, Salucci, Stel.  \emph{Mon. Not. R. Astron. Soc.} 281:27 (1996)
\bibitem{Battaner} Battaner, Florido.  \emph{Fund. Cosmic Phys.} 21:1 (2000)
\bibitem{Binney} Binney, Tremaine. \emph{Galactic Dynamics}. Princeton, NJ: Princeton Univ. Press (1988)
\bibitem{Javorsek} Javorsek II, Fischbach. \emph{The Astrophysical Journal} 568:1-8 (2002)
\bibitem{EROS} EROS Collab.  \emph{Astron. Astrophys.} 400:951 (2003)
\bibitem{Alcock} Alcock, et al. (MACHO collab.)  \emph{Ap. J.} 542:281 (2000)
\bibitem{Tytler} Tytler, et al. \emph{Ap. J.} 617:1-28 (2004)
\bibitem{Spergel} Spergel, et al. \emph{Ap. J. Suppl.} 148:175 (2003)
\bibitem{Robertson} Robertson. \emph{J. Phys.:Conf. Ser.} 173:012016 (2009)
\bibitem{Parke} Parke.  \emph{FERMILAB-CONF-06-248-T} (2006)
\bibitem{Pastor} Pastor.  \emph{Physics of Particles and Nuclei} 42(4):628-640 (2012)
\bibitem{Kolb} Kolb, Turner. \emph{The Early Universe}.  Cambridge, MA: Perseus (1994)
\bibitem{Jungman} Jungman, Kamionkowski, Griest.  \emph{Phys. Rep.} 267:195 (1996)
\bibitem{Goldberg} Goldberg.  \emph{Phys. Rev. Lett.} 50:1419 (1983)
\bibitem{Ellis} Ellis, et al. Nucl. \emph{Phys. B} 238:453 (1984)
\bibitem{Bennett} Bennett, et al. \emph{Ap. J. Suppl.} 148:1 (2003)
\bibitem{Roszkowski} Roszkowski.  \emph{Pramana} 62:1 (2004) hep-ph/0404052
\bibitem{Baluni} Baluni. \emph{Phys. Rev. D} 19, 2227 (1979)
\bibitem{Chung} Chung, et al.  \emph{Phys. Rev. D} 59:023501 (1999) arXiv:hep-ph/9802238
\bibitem{Goodman} Goodman, Witten. \emph{Phys. Rev. D} 31:3059 (1985)
\bibitem{Lewin} Lewin, Smith. \emph{Astropart. Phys.} 6:87 (1996)
\bibitem{Gaitskell} Gaitskell.  \emph{Annu. Rev. Nucl. Part. Sci.} 54:315-359 (2004)
\bibitem{McKinsey} McKinsey, et al.  \emph{Journal of Physics: Conference Series} 203:012026 (2010) 
\bibitem{Fiorucci} Fiorucci, et al.  \emph{AIP Conference Proceedings} 1200:977 (2010)
\bibitem{Kastens} Kastens, et al. \emph{Phys. Rev. C} 80:045809 (2009)

\bibitem{D'Amico} D'Amico, Guido, et al. arXiv:0907.1912[astro-ph.CP] 
\bibitem{Wu} Wu, et al.  \emph{Monthly Notices of the Royal Astronomical Society} 301(3):861-971.  arXiv:astro-ph/9808179
\bibitem{Griest} Griest, et al.  \emph{Phys. Rept.} 333:167-182 (2000) 
\bibitem{Aprile} E. Aprile, T.Doke. \emph{Rev. Mod. Phys.} 82:2053-2097 (2010)



\end{thebibliography}


\end{document}

