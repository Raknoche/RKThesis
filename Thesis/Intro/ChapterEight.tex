\section{Beyond LUX} \label{LZChapter}

\subsection{Future LUX Analyses}

The LUX experiment finished collecting WIMP search data in September 2016.  The results of the spin-independent dark matter search will be submitted to Physical Review Letters in the coming months, opening new directions for data analysis and science results.  In particular, the existing data can be used to search for exotic dark matter interactions, such as the inelastic scattering of dark matter on atomic nuclei.  

The initial LUX analysis was optimized to search for the elastic scattering of dark matter with recoil energies less than 30 keV.  This low energy cut-off hinders the search for most exotic dark matter interactions, and much of the future analysis will require a measurement of the high energy nuclear and electron recoil yields in liquid xenon.  A $^{14}$C methane calibration source, which was designed based on the work in Chapter~\ref{TritiumChapter}, has been injected in an end-of-LUX calibration campaign and can provide the latter half of these yield measurements.


\subsection{Signal Corrections in the LZ Detector}

To continue the search for dark matter, a new direct detection experiment known as LZ is under construction.  The detector is designed for a WIMP-nucleon cross section sensitivity of $\sim 2 \times 10^{-48}$ cm$^2$, improving the world leading limits set by LUX by two orders of magnitude.  The LZ design has been embraced by the dark matter community and the US funding agencies, and is scheduled to collect data in 2020.

The $^{83m}$Kr calibration source used to produce signal corrections in Chapter~\ref{StandardCalibrations} has a short 1.86 hour half life that may prevent uniform mixing throughout LZ's fiducial volume.  To circumvent this issue, the LZ collaboration is developing a $^{131m}$Xe calibration source.  The 164 keV mono-energetic decay of $^{131m}$Xe, as well as its longer 11.9 day half life, allow the methods of Chapter~\ref{StandardCalibrations} to be applied, resulting in pulse area corrections for the entire LZ fiducial volume.  However, the higher energy of the decay will saturate the LZ PMT arrays, and introduce additional systematic uncertainties if the LZ drift field is non-uniform. 

Although a lack of natural mixing would make $^{83m}$Kr undesirable as a weekly calibration source, a thermal gradient could be applied with the detector's heaters to induce convection in a one time $^{83m}$Kr calibration campaign.  This would allow measurement of the saturation effects in $^{131m}$Xe data via comparison to contemporaneous $^{83m}$Kr data, and provide direct measurement of the drift field throughout the detector via the techniques discussed in Chapter~\ref{Run04Corrections}. 


\subsection{Calibrations of the LZ Detector}

The CH$_3$T calibration source described in Chapter~\ref{TritiumChapter} will be used to measure the electron recoil response of LZ.  Similarly, the energy scale calibrations discussed in Chapter~\ref{CombinedEnergyModel} are directly applicable to LZ.  The mono-energetic peaks from activation lines, as well as the $^{83m}$Kr and $^{131m}$Xe calibration sources, can produce an LZ Doke plot, and the tritiated methane and $^{14}$C methane sources can confirm the gain measurements.

