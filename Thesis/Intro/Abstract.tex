\thispagestyle{plain}
\pagenumbering{gobble}
\begin{center}
    \Large
    \textbf{ABSTRACT}
    
    \vspace{1cm}
    \large
    \begin{tabular}{ll}
  	Title of dissertation:& SIGNAL CORRECTIONS AND CALIBRATIONS \\[-0.5cm]
  	& IN THE LUX DARK MATTER DETECTOR \\[0.3cm]
  	& Richard Knoche, Doctor of Philosophy, 2016 \\[0.2cm]
    Dissertation directed by:& Professor Carter Hall \\[-0.5cm]
    & Department of Physics
 	\end{tabular}

	\vspace{1cm}
    \normalsize

\end{center}
    The Large Underground Xenon (LUX) Detector has recently finished a 332-day exposure and placed world-leading limits on the spin-independent WIMP-nucleon scattering cross section.  In this work, we discuss the basic techniques to produce signal corrections, energy scale calibrations, and recoil band calibrations in a dark matter detector.  We discuss a nonuniform electric field that was present during LUX's 332-day exposure, and detail how such a field complicates these calibration techniques.  Finally, we expand on the techniques presented earlier, such that they account for all of the complication introduced by the nonuniform electric field, and allow a limit to be produced from the data.

